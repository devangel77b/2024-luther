\documentclass[12pt]{wrceletter}


\name{Dennis Evangelista}
\position{Assistant Professor}
%\email{\href{mailto:evangeli@usna.edu}{\emph{evangeli@usna.edu}}}
\email{evangeli@usna.edu}
\telephone{410-293-6132}

\date{\today}

\usepackage{designature}
\signature{\vspace*{-0.7in}\includesignature\\Dennis Evangelista}
%\signature{Dennis Evangelista\\Assistant Professor} % title not needed if in letterhead
\address{\null} %{105 Maryland Avenue\\Annapolis, MD 21402} % leave blank, provided in letterhead
%\longindentation=0in % to change signature to be flushleft

\begin{document}
\begin{letter}{% recipient address here
}

% opening here
\opening{Dear WHOEVER:}
\raggedright % if you like this sort of thing
\setlength{\parindent}{15pt} % if you like this sort of thing


I am happy to recommend Ms. Angeline Luther for your program. I worked with Angeline over the summer of 2019 when she was a high school summer intern in the Department of Electrical and Computer Engineering at the US Naval Academy. In addition to her work with Dr.~Coxson in ECE, Angeline worked with me on a biomechanics project examining the effect of ear motion on sonar reception and processing in horseshoe bats (\emph{Rhinolophus sp.}). 

Angeline was new to biomechanics but brought great enthusiasm and resourcefulness in learning about the subject and developing sketch models of a robotic pair of bat ears. She was also quite interested in the deeper computer science, engineering design, and biology principles involved in the work. The ultimate goal of the robotic ear system is to test hypotheses of the effects ear motion has on Doppler shifts, spatial resolution, and target classification. Angeline's work is forming the basis for a future team of midshipmen to pick up where she left off; in short, she can only be replaced by four or five engineering seniors. 

Angeline is early in her studies but rapidly picked up C and C++ programming on both Arduino and mbed LPC1768 microprocessors. In the space of a few days she reached a level of proficiency that takes our USNA midshipman months to reach in the Principles of Mechatronics class for sophomores. I have no doubt as to her continued success in whatever projects she takes on in the future, at whatever high-powered school she ends up at next. 

Angeline is an outstanding candidate! I am excited to see what she does next!

%-I worked with you over the summer on a bat biosonar project involved with learning and researching about horseshoe bats, and then taking that knowledge to build a bat ear model
%-I came in with very limited knowledge on the subject, and was able to gain a much deeper understanding by researching and using resources that were available to me (such as your knowledge, online videos and resources such as JHU, etc)
%-I was excited and invested in the project, and enjoyed using programming and applying it to the project
%-Over the course of the summer, my model drastically improved, and I continued to make improvements, taking the bat ear from simple rigid movements, to using timers and various functions to add more capabilities to the ear model
%-I was able to apply knowledge from classes such as physics (sound, Doppler effect) and biology and combine it with programming
%You definitely have more knowledge than me in terms of writing recommendation letters and I'm not sure what is typically in a recommendation letter, so those were just a few ideas I had.  I'll let you know asap when I need recommendations.  Thanks for letting me work with you, and good luck on the continuation of the project!


\closing{Very respectfully,} % provides empty 

%\ps{post script here}
%\encl{enclosure here}
\end{letter}
\end{document}